%%This is a very basic article template.
%%There is just one section and two subsections.

%%******************************************************************************
%% FRONT PAGE
%%******************************************************************************

% \thispagestyle{empty}
% 
% %% Restart page counter.
% \setpagecounter{0}
% 
% %% Disable page anchor to avoid multiple page number definition warnings.
% \hypersetup{pageanchor=false}
% 
% \vspace{4cm}
% 
%  \textcolor{gray}{Execução:} \

%%==============================================================================




\documentclass[a4size] {article} 
\usepackage{graphicx}
\usepackage[utf8]{inputenc}
\usepackage{float}
\usepackage {subcaption}

\usepackage{color}







\begin{document}

\include{front_page}

% \title{Relatório de Viagem}
% \author{Julia Campana, Eduardo Elael}
% \date{27/04/15 - 01/05/15}
% \maketitle


% \begin{flushright}
%   \emph{Versão:} 07/maio/2014
% \end{flushright}



\section*{Introdução}
A viagem a Jirau realizada entre os dias 27 de Abril e 01 de Maio teve como
objetivo coletar informações para o a fase conceitual do Projeto EMMA. 

\section*{Objetivo de Viagem}
\begin{itemize}
  \item Breve apresentação SOTA (Estado da Arte de sistema robóticos
  seme\-lhantes);
  \item Alinhamento técnica com Rijeza, perguntas e respostas relacionadas ao
  processo de hardcoating (descritas no artigo);
\item Definição das tarefas do robô EMMA com a ESBR;
\item Discussão do problema relacionadas ao ambiente da turbina (descritas no
artigo) 
\item Demonstração do processo de hardcoating pela Rijeza;
\item Visita à uma turbina para inspecionar os pontos de acesso.;
\end{itemize}


\section*{Execução}
Em um primeiro momento foi realizada a reunião para a apresentação de estado da
arte de sistemas robóticos que executam tarefas semelhantes às que serão
realizadas, seguida de perguntas e respostas aos técnicos da Rijeza e ESBR para
esclarecimento dos aspectos técnicos pertinentes ao projeto.

\begin{figure}[H]
\centering
\begin{subfigure}[b]{0.46\textwidth}
\includegraphics[height=3.2cm]{Fotos/img_4836.jpg}
\caption{Reunião com Rijeza e ESBR.}
\label{fig:gull}
\end{subfigure}%
\quad
\begin{subfigure}[b]{0.46\textwidth}
\includegraphics[height=3.2cm]{Fotos/img_4881.jpg}
\caption{Equipe visita o galpão da Rijeza.}
\label{fig:tiger}
\end{subfigure}
\end{figure}

% \begin{figure}[H]
% \centering
% \includegraphics[width=0.85\textwidth]{Fotos/img_4836.jpg}
% \caption{Reunião com Rijeza e ESBR.}
% \end{figure}

Posteriormente, visitamos as instalações da Rijeza dentro da usina para
presenciar o funcionamento do robô que executa o \textit{hardcoating}, assim como o
estado das pás onde o \textit{hardcoating} já havia sido aplicado. Visando compreender
melhor o processo de metalização, suas etapas e especificidades quando realizado
sobre uma pá da turbina.

\begin{figure}[H]
\centering
\begin{subfigure}[b]{0.46\textwidth}
\includegraphics[height=3.2cm]{Fotos/img_4881.jpg}
\caption{Equipe do projeto EMMA visita o galpão da Rijeza para verificar o
processo de inspeção vigente.}
\label{fig:gull}
\end{subfigure}%
\quad
\begin{subfigure}[b]{0.46\textwidth}
\includegraphics[height=3.2cm]{Fotos/img_4858.jpg}
\caption{Instalações da Rijeza: Robô Industrial utilizado para aplicação de hardcoating.}
\end{subfigure}
\end{figure}

% \begin{figure}[H]
% \centering
% \includegraphics[width=0.85\textwidth]{Fotos/img_4881.jpg}
% \caption{Equipe visita o galpão da Rijeza.}
% \end{figure}
% 
% \begin{figure}[H]
% \centering
% \includegraphics[width=0.85\textwidth]{Fotos/img_4858.jpg}
% \caption{Instalações da Rijeza: Robô Industrial utilizado para
% aplicação de hardcoating.}
% \end{figure}

% -----------------------------------------------------------------

\begin{figure}[H]
\centering
\begin{subfigure}[b]{0.46\textwidth}
\includegraphics[height=3.2cm]{Fotos/img_4850.jpg}
\caption{Instalações da Rijeza: secador, dosador, cilindros de gases, medidores,unidade de segurança e o console do robô.}
\label{fig:gull}
\end{subfigure}%
\quad
\begin{subfigure}[b]{0.46\textwidth}
\includegraphics[height=3.2cm]{Fotos/img_4852.jpg}
\caption{Instalações da Rijeza: Maquinário utilizado, equipamentos como cooler e
compressor ao fundo do galpão onde inspeção é realizada.}
\end{subfigure}
\end{figure}

% \begin{figure}[H]
% \centering
% \includegraphics[width=0.85\textwidth]{Fotos/img_4850.jpg}
% \caption{Instalações da Rijeza: secador, dosador, cilindros de gases, medidores,
% unidade de segurança e o console do robô.}
% \end{figure}
% 
% \begin{figure}[H]
% \centering
% \includegraphics[width=0.85\textwidth]{Fotos/img_4852.jpg}
% \caption{Instalações da Rijeza: cooler e compressor ao fundo.}
% \end{figure}

% -----------------------------------------------------------------

Na parte da tarde visitamos uma unidade geradora para nos familiarizamos com o
local onde a aplicação de hardcoating será feita, nosso objetivo foi inspecionar
os locais de acesso, a montagem dos andaimes e a geometria da hélice.

% -----------------------------------------------------------------

\begin{figure}[H]
\centering
\begin{subfigure}[b]{0.46\textwidth}
\includegraphics[height=3.2cm]{Fotos/img_4886.jpg}
\caption{Pá já com Hardcoating aplicado.}
\label{fig:gull}
\end{subfigure}%
\quad
\begin{subfigure}[b]{0.46\textwidth}
\includegraphics[height=3.2cm]{Fotos/img_4905.jpg}
\caption{Acesso a unidade geradora.}
\end{subfigure}
\end{figure}


% \begin{figure}[H]
% \centering
% \includegraphics[width=0.85\textwidth]{Fotos/img_4886.jpg}
% \caption{Pá já com Hardcoating aplicado.}
% \end{figure}
% 
% \begin{figure}[H]
% \centering
% \includegraphics[width=0.85\textwidth]{Fotos/img_4905.jpg}
% \caption{Corredor de acesso unidade geradora.}
% \end{figure}

% -----------------------------------------------------------------






\begin{figure}[H]
\centering
\includegraphics[width=0.85\textwidth]{Fotos/img_4931.jpg}
\caption{Montagem de andaimes dentro da unidade geradora.}
\end{figure}

\begin{figure}[H]
\centering
\includegraphics[width=0.85\textwidth]{Fotos/img_4966.jpg}
\caption{Engenheiros medem hélices.}
\end{figure}

% -----------------------------------------------------------------
\begin{figure}[H]
\centering
\includegraphics[width=0.85\textwidth]{Fotos/img_4978.jpg}
\caption{Em detalhe a escoltilha de acesso superior.}
\end{figure}

O segundo dia em Jirau foi voltado para a formulação dos conceitos possíveis
para o Robô EMMA. Em um primeiro momento avaliamos os braços mecânicos
disponíveis no mercado e todas suas possíveis aplicações no espaço da unidade
gerador onde a inspeção deve acontecer. Em seguida, levantamos hipóteses sobre
as possíveis vias de entradas e suas implicações. 

\section*{Resultado}
A viagem se mostrou de grande relevância para a viabilidade técnica do projeto,
uma vez que nos deu parâmetros para formular os conceitos que serão explorados
como soluções para a inspeção de turbinas. Realizamos as reuniões
necessárias, a inspeção da unidade geradora, bem como conhecemos as instalações
da Rijeza onde o hardcoating é feita. No ultimo dia organizamos tais informações
para que sejam devidamente adicionas em nosso Relatório Técnico.

\end{document}
